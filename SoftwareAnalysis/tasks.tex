\subsection{Tasks and scheduling}
\label{subsec:tasks}
\paragraph{Tasks} In a CRBASIC program, main scan, secondary scans and measurement processes
are identified as Task.
Every task has its own priority and it is arranged in one of three scheduling queues ordered by priority:
\begin{enumerate}
	\item Measurement tasks: have the maximum priority because they are considered like real-time processes.
	\item SDM tasks: device management tasks issued by the CS OS.
	\item Processing Tasks: all other tasks.
\end{enumerate}
\paragraph{CPU execution mode}
Data-logger's program can compiled in Sequential Mode or Pipeline 
(the choice is done by the compiler itself or can be forced using a directive in the program).
Within the first mode instructions are executed in the order they are written in the program.
This means there is no reorder of instruction in the CPU, as it happens in Pipeline mode.
\paragraph{Main and secondary scans}
Among all the scans, the one identified as "Main Scan" has the higher priority.
All the other scans can execute once the main scan has finished its internal tasks,
or when the main scan is not active. This means that in Pipeline mode,a single scan
can be split over multiple executions of the main scan.
