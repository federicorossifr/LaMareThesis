\subsection{CRBASIC program structure}
\paragraph{}Datalogger's program can be split in three parts:
\begin{itemize}
	\item Main scan (or main loop)
	\item Secondary scans (also called slow sequence scans)
	\item Data table declarations (plus functions and global variables)
\end{itemize}
\paragraph{Main Scan}
The main scan is located at the first {\tt Scan} instruction after the {\tt BeginProg} directive.
It contains data measurement and processing along with {\tt CallTable <TableName>} instructions
that saves data into the table specified.
\paragraph{Secondary scans}
Secondary scans,also called slow sequence scans, 
are ordinary loops marked with the keyword {\tt SlowSequence}.
In \ref{subsec:tasks} more informations about scan and task scheduling are provided.
\paragraph{Data table}
Data tables are declared with the {\tt DataTable} directive. It takes table name and
trigger variable (in the considered case trigger is always set to true) as arguments.
The size can be specified, but specifing an auto-sizing table (size=0) has to be preferred.
\paragraph{} Table columns are enclosed between {\tt DataTable} and {\tt EndTable} directives, together with the "DataInterval"
directive, that specifies on which time basis the table has to be saved for output
and the {\tt FieldNames} directive, that specifies column names.
\paragraph{}Columns are declared by specifing the statistic function 
to apply to data from {\tt CallTable} directives (e.g Average,Sample,Minimum)
and the variable from where desired value will be fetched during {\tt CallTable}
\paragraph{} A particular directive, called {\tt WindVector}, allows to store
data from pulse sensors like the wind vector used in this scenario.