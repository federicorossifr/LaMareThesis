\subsection{Data Table Definition}
The following piece of code shows the definition of a data table.
As underlined by {\tt DataInterval} directive, the first data table is stored every 15 minutes, while the second is stored every 60 minutes.
\begin{lstlisting}[numbers=left, breaklines=true]
DataTable(Table1,True,-1)
	DataInterval(0,15,Min,10)
	Average(1,RGup,FP2,False)
	Average(1,RGdown,FP2,False)
	Average(1,AirTC,FP2,False)
	Average(1,RH,FP2,False)
	WindVector (1,Vvento,DirVento,FP2,False,0,0,2)
	FieldNames("Vvento_S_WVT,Vvento_U_WVT,DirVento_DU_WVT
	           ,DirVento_SDU_WVT")
	Average(1,IRup,FP2,False)
	Average(1,IRdown,FP2,False)
	Average(1,IRupc,FP2,False)
	Average(1,IRdownc,FP2,False)
	Average(1,T107_1,FP2,False)
	Average(1,T107_2,FP2,False)
	Average(1,Thmp45,IEEE4,0)
	Average(1,URhmp45,IEEE4,0)
	Sample(1,Status.SW12Volts(1,1),Boolean)
EndTable

DataTable(Table2,True,-1)
	DataInterval(0,60,Min,10)
	Minimum(1,Batt_Volt,FP2,False,False)
	Sample(1,DT,FP2)
EndTable
\end{lstlisting}
\clearpage
\subsection{Swithing modem on}
The following code shows how the hysteresis control is applied to decide whether the modem can be swtiched on or not.
\begin{lstlisting}[numbers=left, breaklines=true]
Dim lowBatTh As Float = 11.50
Dim highBatTh As Float = 13.20

Sub PowerOn
battPowerOnVoltage = Batt_Volt
 If (battPowerOnVoltage < lowBatTh) Then
  powerOnFlag = 0
  Else If (battPowerOnVoltage > highBatTh OR 
          ((battPowerOnVoltage >= lowBatTh) AND powerOnFlag)) Then
   SW12(1)
   powerOnFlag = 1
 EndIf
EndSub
\end{lstlisting}
\subsection{Retrieving data from sensors}
The following code shows the instructions given to the datalogger in order to retrieve data from sensors. Every 60 seconds (specified by the instruction {\tt Scan} the datalogger read measurements from sensors and store them in data table with the instruction {\tt CallTable}
\begin{lstlisting}[numbers=left, breaklines=true]
Scan(60,Sec,1,0)
    RealTime(rTime) 

    VoltSe(Thmp45,1,mV2500,13,0,0,_50Hz,0.1,-40)
    VoltSe(URhmp45,1,mV2500,14,0,0,_50Hz,0.1,0)
    If URhmp45 >100 Then URhmp45=100
    
    Battery(Batt_Volt)    

    'Pyranometer measurements Solar_kJ and RGup:
		VoltDiff(RGup,1,mv25,1,True,0,_50Hz,1,0)
		If RGup<0 Then RGup=0
		Solar_kJ=RGup*7.46268
		RGup=RGup*124.378
		
    'Pyranometer measurements Solar_k_2 and RGdown:
		VoltDiff(RGdown,1,mv25,2,True,0,_50Hz,1,0)
		If RGdown<0 Then RGdown=0
		Solar_k_2=RGdown*7.46268
		RGdown=RGdown*124.378
		
    'CS215 Temperature & Relative Humidity Sensor 
		SDI12Recorder(AirTC,1,"0","M!",1,0)
		
    
    '05103 Wind Speed & Direction Sensor
		PulseCount(Vvento,1,1,1,1,0.098,0)
		BrHalf(DirVento,1,mV2500,5,1,1,2500,True,0,_50Hz,355,0)
		If DirVento>=360 Then DirVento=0
		
    'Every 60 minutes read Sonic Ranging Sensor
    If TimeIntoInterval(0,60,Min) Then
    'SR50 Sonic Ranging Sensor
		SDI12Recorder(DT,7,"0","M!",100.0,0)
		TCDT=DT*SQR((AirTC+273.15)/273.15)
		EndIf
    
    'Wiring Panel Temperature measurement TCR1000:
		PanelTemp(TCR1000,_50Hz)
		
    'Generic Differential Voltage measurements IRup:
		VoltDiff(IRup,1,mV25,4,True,0,_50Hz,86.881,0.0)
		'Generic Differential Voltage measurements IRdown:
		VoltDiff(IRdown,1,mV25,5,True,0,_50Hz,118.2,0.0)
		
    '/*****************************/
    .......
    '/*****************************/
   	
    'Call Data Tables and Store Data
		CallTable(Table1)
		CallTable(Table2)
NextScan
\end{lstlisting}
\subsection{Modem-on windows}
\paragraph{}Following code shows the part of the program that has to edited in order to modify modem-on windows in
datalogger.
\begin{lstlisting}[numbers=left, breaklines=true]
=====> Number of daily modem-on windows
Const NumWindows = 3 
/......./
'Set true to days you want the modem to be switched on
Monday = true
Tuesday = true
Wednesday = true
Thursday = true
Friday = true
Saturday = true
Sunday = true

=====> First daily window
WHourStart(1) = 10 'Ora di inizio della Prima Finestra (0-23)
WMinStart(1) = 31 'Minuto di inizio della Prima Finestra (0-59)
WNumMin(1) = 30 'Durata in minuti della Prima Finestra

=====> Second daily window
WHourStart(2) = 13 'Ora di inizio della Seconda Finestra (0-23)
WMinStart(2) = 31 'Minuto di inizio della Seconda Finestra (0-59)
WNumMin(2) = 30 'Durata in minuti della Seconda Finestra

=====> Third daily window
WHourStart(3) = 20 'Ora di inizio della Terza Finestra (0-23)
WMinStart(3) = 50 'Minuto di inizio della Terza Finestra (0-59)
WNumMin(3) = 0 'Durata in minuti della Terza Finestra
\end{lstlisting}