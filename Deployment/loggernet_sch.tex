\subsection{Scheduling data collection operations from Loggernet}
Scheduling data collections can be accomplished by synchronizing modem-on windows on the datalogger
and schedules on remote Loggernet. Note that datalogger has clock set on GMT+1, while Loggernet is set on
GMT.
\paragraph{Set modem-on windows on datalogger}
In order to set modem-on windows on datalogger, the program running on CR1000 has to be edited and re-compiled
on the datalogger. See appendix \ref{appendix:fcode} to find more information on the code to be edited.
\paragraph{Set Loggernet's schedules}
To configure Loggernet's schedules:
\begin{enumerate}
	\item Open remote desktop connection as shown in section \ref{subsec:rdp} and start Loggernet if stopped.
	\item Click on {\tt Set-up} in {\tt Main} tab.
	\item Select {\tt CR1000} datalogger from the panel on the left.
	\item On tab {\tt Schedule} check {\tt Scheduled Collection Enabled}
	\item Set {\tt Time} in order to match one of the modem-on windows just configured on datalogger. NOTE: set the time 5 or more minutes after
		  modem-on windows start to permit modem to register on the network.
	\item Set {\tt Collection interval} to any value in days. (Typically use 3 days interval between collection)
	\item Set {\tt Primary retry interval} at {\tt 10m} and {\tt Number of primary retries} at {\tt 1}
	\item Check {\tt Stay on collect schedule}.
	\item On the drop-down below select {\tt Automatically reset changed tables} and check {\tt Enable automatic hole collection}
	\item Once finished, click on {\tt Apply} at the bottom of {\tt Setup} window.
\end{enumerate}
\paragraph{Schedule FTP task}
Once data collection schedule is set, create a task that forwards data to a remote FTP server.
\begin{enumerate}
	\item Open remote desktop connection as shown in section \ref{subsec:rdp} and start Loggernet if stopped.
	\item Click on {\tt Task master} on {\tt Main} tab.
	\item Click on {\tt CR1000} under {\tt Tasks}, then click on the button {\tt Add after} below.
	\item On the right, set {\tt Station Event Type} to {\tt After file closed}.
	\item Click on {\tt Configure task...} below, a new window will be opened.
	\item On tab {\tt FTP Settings} check {\tt Enable FTP}. Set following parameters according to appendix \ref{appendix:cred} (service FTP SERVER). Browse {\tt Remote folder} with button on the right and select preferred folder where data files will be uploaded. Confirm with {\tt OK}
	\item Once finished, click on {\tt Apply} at the bottom of {\tt Task master} windows.
\end{enumerate}
