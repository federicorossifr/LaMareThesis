\subsection{Configuring device into local Loggernet}
\paragraph{}
Connection in local environment is accomplished by a serial RS232 serial connection.
\paragraph{Serial link requirements}
To connect device via serial link user has to install a driver for the serial-to-USB adapter, that can be downloaded from \cite{prol1}.
\paragraph{Adding device to Loggernet}
Once the device is connected, user can now open Loggernet and click on {\tt Set-up} in {\tt Main} tab. If it's the first time running Loggernet, it will open the {\tt EZSetup Wizard}, 
otherwise user has to click on {\tt Add} button located in the tool-bar.
\begin{enumerate}
	\item In {\tt Communication Set-up} section choose CR1000 and go to the next step.
	\item Then choose connection type ({\tt Direct} if serial or {\tt Phone Modem} if Satellite)
	\item Then follow instructions on program to find the correct COM port (and satellite SIM number if using satellite connection) to connect with
	\item Skip {\tt Datalogger settings}  by pressing Next, if using standard parameters (potentially set baud rate at {\tt 9600}).
	\item Check selected parameters in {\tt Set-up summary}
	\item Verify parameters in {\tt Communication Test}
	\item If test passes press {\tt Finish}
\end{enumerate}
\paragraph{}
Added device can now be seen by pressing {\tt Set-up} in {\tt Main} tab.

\subsection{Configuring device into remote Loggernet}
\paragraph{}
Connection from remote environment is accomplished by using the satellite connection infrastructure explained in section \ref{subsec:loggernet}.
Locally, connect CR1000 to modem as shown in section \ref{subsec:datacable}.
\paragraph{}
To add a device to remote Loggernet:
\begin{enumerate}
	\item Open remote desktop connection as shown in section \ref{subsec:rdp} and start Loggernet if stopped.
	\item Click on {\tt Set-up} in {\tt Main} tab.
	\item Click on {\tt Add Root} on the tool-bar and select {\tt ComPort}.
	\item In the {\tt Network map} below, click on the just added ComPort and on the right select in {\tt ComPort Connection} {\tt FabulaTech Serial Port Redirector}. Below set {\tt Delay Hangup} at {\tt 500ms} and {\tt Communication Delay} at {\tt 2s}.
	\item In the {\tt Network map} right-click on the ComPort and select {\tt PhoneBase}.
	\item Click on the PhoneBase added and set {\tt Maximum baud rate} at {\tt 9600}, {\tt Response Time} at {\tt 2s} and {\tt Delay Hangup} at {\tt 500ms}.
	\item Right-click on the PhoneBase and select {\tt PhoneRemote}.
	\item Click on the PhoneRemote added and add the SIM phone number of the CR1000 modem with a delay of {\tt 500ms}.
	\item Right-click on PhoneRemote and select {\tt Generic}.
	\item Click on the Generic added and set {\tt Baud rate} at {\tt 9600}, {\tt Response time} at {\tt 4s}, {\tt Maximum packet size} at {\tt 2048} and {\tt Delay Hangup} at {\tt 2s 500ms}. In tab {\tt Modem} set {\tt Dial string} to {\tt D10000}.
	\item Right-click on the Generic and select {\tt PackBusPort}
	\item --- to be continued
\end{enumerate}