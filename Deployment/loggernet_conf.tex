\subsection{Configuring device into local Loggernet (or PC200W)}
\paragraph{}
Connection in local environment is accomplished by a serial RS232 serial connection. While PC200W is distributed as free software, Loggernet is offered both in a paid version and a 30-day trial version. PC200W can be downloaded from \cite{cs1}. Loggernet trial version can be downloaded from \cite{cs2}
\paragraph{Serial link requirements}
To connect device via serial link user has to install a driver for the serial-to-USB adapter, that can be downloaded from \cite{prol1}. 
\paragraph{Adding device}
Once the device is connected, user can now open Loggernet and click on {\tt Set-up} in {\tt Main} tab. If it's the first time running Loggernet (or PC200W), it will open the {\tt EZSetup Wizard}, 
otherwise user has to click on {\tt Add} button located in the tool-bar (if using Loggernet, make sure Set-up is running in simplified mode by checking it in the title bar. If not, click on the button {\tt EZ View} on the toolbar).
\begin{enumerate}
	\item In {\tt Communication Set-up} section choose CR1000 and go to the next step.
	\item Then choose connection type {\tt Direct}
	\item Select COM port to connect with (check Windows' Device Manager to discover COM port, it should display {\tt Prolific USB to Serial adapter}).
	\item Set {\tt Baud rate} at {\tt 9600} and {\tt PakBus Address} at {\tt 1}.
	\item Skip the following {\tt Datalogger settings} by pressing Next if not using any type of encryption.
	\item Check choosen parameters in {\tt Set-up summary}
	\item Verify parameters in {\tt Communication Test} by clicking {\tt Next}
	\item If test passes, in the following tab set the datalogger clock by clicking {\tt Check Datalogger Clock}
	\item Select and send a program to the datalogger by clicking on {\tt Select and Send program}
	\item \emph{Only for Loggernet:} click on {\tt Get table definition} to retrieve table structures from the datalogger as defined in the program
	\item \emph{Only for Loggernet:} enable {\tt Scheduled collection }  and set the time  and interval for automatic data collection.
	\item Click {\tt Finish} to close the wizard.
\end{enumerate}
\paragraph{}
Added device can now be seen by pressing {\tt Set-up} in {\tt Main} tab.

\subsection{Configuring device into remote Loggernet}
\paragraph{}
Connection from remote environment is accomplished by using the satellite connection infrastructure explained in section \ref{subsec:loggernet}.
Locally, connect CR1000 to modem as shown in section \ref{subsec:datacable}.
\paragraph{}
To add a device to remote Loggernet:
\begin{enumerate}
	\item Open remote desktop connection as shown in section \ref{subsec:rdp} and start Loggernet if stopped.
	\item Click on {\tt Set-up} in {\tt Main} tab.
	\item Click on {\tt Add Root} on the tool-bar and select {\tt ComPort}.
	\item In the {\tt Network map} below, click on the just added ComPort and on the right select in {\tt ComPort Connection} {\tt FabulaTech Serial Port Redirector}. Below set {\tt Delay Hangup} at {\tt 500ms} and {\tt Communication Delay} at {\tt 2s}.
	\item In the {\tt Network map} right-click on the ComPort and select {\tt PhoneBase}.
	\item Click on the PhoneBase added and set {\tt Maximum baud rate} at {\tt 9600}, {\tt Response Time} at {\tt 2s} and {\tt Delay Hangup} at {\tt 500ms}.
	\item Right-click on the PhoneBase and select {\tt PhoneRemote}.
	\item Click on the PhoneRemote added and add the SIM phone number of the CR1000 modem with a delay of {\tt 500ms}.
	\item Right-click on PhoneRemote and select {\tt Generic}.
	\item Click on the Generic added and set {\tt Baud rate} at {\tt 9600}, {\tt Response time} at {\tt 4s}, {\tt Maximum packet size} at {\tt 2048} and {\tt Delay Hangup} at {\tt 2s 500ms}. In tab {\tt Modem} set {\tt Dial string} to {\tt D10000}.
	\item Right-click on the Generic and select {\tt PackBusPort}
	\item Click on the added PackBusPort and set {\tt Maximum Time On-Line} at {\tt 10m}, {\tt Baud Rate} at {\tt 9600}, {\tt Beacon Interval} at {\tt 1m}, {\tt Extra response time} at {\tt 4s} and {\tt Delay Hangup} at {\tt 2s 500ms}.
	\item Righ-click on the PackBusPort and select {\tt CR1000}
	\item Click on the CR1000 and set {\tt PakBus Address} at {\tt 1}, {\tt Packet Size} at {\tt 1000} and {\tt Delay Hangup} at {\tt 2s 500ms}.
\end{enumerate}