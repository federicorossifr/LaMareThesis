\subsection{Deploying GSN on Linux-Windows host}
\paragraph{Prerequisites} GSN module has the following dependencies:
\begin{itemize}
	\item Java: JDK and JRE.
	\item Apache Ant for source compilation.
	\item MySQL Server.
\end{itemize}
\paragraph{Source code} Source code provided is a custom version of original GSN 
source code available at \cite{epfl1}
\paragraph{Preliminary Configuration} Open a terminal on the root folder and execute the {\tt init.sh} script. It will create
 default user and database used by the application. The default configuration can be found in {\tt gsn\_dev/conf/gsn\_conf.xml}
\begin{lstlisting}[numbers=left, breaklines=true]
<storage user="gsn" password="gsnpassword" driver="com.mysql.jdbc.Driver" url="jdbc:mysql://localhost/gsn" />
\end{lstlisting}
\paragraph{Schedule FTP fetch} Open a terminal on the root folder and execute the command {\tt crontab -u your\_username cronfile}. It will schedule the FTP data fetch from LaMare FTP server. At this point everything is configured.
\paragraph{Running GSN} Open a command prompt on the project root {\tt gsn-dev} and execute 
{\tt ant gsn} to compile and execute the application. Test the application in a web browser at the link
{\tt http://127.0.0.1:22001}. You should see the main page showing pre-configured virtual sensors.